\documentclass[fontsize=10pt,paper=a4,bibliography=totoc]{scrartcl}

%\usepackage[utf8]{inputenc}
%\usepackage[ngerman]{babel}
%\usepackage{amsmath}
%\usepackage{graphicx}
%\usepackage{units}

% packages  
\usepackage[utf8]{inputenc}
\usepackage[T1]{fontenc}
\usepackage{ae}
\usepackage{tabularx}
\usepackage{amsmath,amssymb}
\usepackage[pdftex]{graphicx}
\usepackage{subfigure}
\usepackage{url}
\usepackage[T1]{fontenc}
\usepackage{ifthen}
\usepackage{hyperref}
\usepackage[absolute,overlay]{textpos}
\usepackage{tikz}
\usepackage[ngerman]{babel}
\usepackage{units}

\graphicspath{{images/}}

\usepackage{url}
\title{Ausarbeitung\\Solarthermisches Kraftwerk}

\author{K. Franke, M. Dolgov, F. Achilles\\The Raymen}
\begin{document}
\maketitle

\section{Stand der Technik}

\subsection{Dish-Stirling Anlage}
Die Dish-Stirling Anlage besteht aus einem großen Parabolspiegel, der meistens aus mehreren kleinen Spiegeln zusammengesetzt wird. Im Brennpunkt befindet sich ein Stirling Motor, der aus der gebündelten Wärme Strom erzeugt. Damit die Strahlen sich immer im Brennpunkt treffen, wird der große Parabolspiegel mit der Sonne mitgedreht um immer senkrecht einfallende Strahlen zu erreichen. Für dieses Konzept ist somit immer eine zweiachsige Nachführung des Spiegels nötig, da nicht senkrecht einfallende Strahlen den Absorber des Stirling Motors verfehlen würden. Dieses Konzept funktioniert daher nur bei direkter Sonneneinstrahlung und nicht bei diffusem Licht. Vorteil dieses Kraftwerktyps ist die sehr kompakte Bauweise, so können diese Kraftwerke schon für Leistungen ab ??
%TODO Finden
 kW eingesetzt werden. Nachteil ist die Größe des großen Spiegels, der ja nach Sonnenstand wie ein Segel wirkt. Je nach Standort muss die Konstruktion für unterschiedlich starke Windlasten ausgelegt werden, die ein solches Projekt teuer machen. 
 
\subsection{Parabolrinnenkraftwerk}
\begin{figure}[htb]
	\centering
	\def\svgwidth{.6\textwidth}
	\includegraphics[width=\textwidth]{Solar_Plant_kl}
	%http://upload.wikimedia.org/wikipedia/commons/6/63/Solar_Plant_kl.jpg 
	\label{pic:parabolrinnen}
\end{figure}
Bei Parabolrinnenkraftwerken wird das einfallende Sonnenlicht von gewölbten Spiegeln auf eine Linie fokussiert. Ein mit Öl gefülltes Absorberrohr verläuft entlang dieser Linie. Die durch das Öl transportierte Energie kann in einem Gas- und Dampfkraftwerk zur Stromerzeugung genutzt werden (s. Abb~\ref{pic:parabolrinnen}). Der Vorteil von Parabolrinnenkraftwerken gegenüber Dish-Stirling-Anlagen besteht darin, dass die Nachführung nur einachsig erfolgen muss und somit ein günstigeres System realisiert werden kann. Ein weiterer Vorteil ist, dass ein stationäres Kraftwerk verwendet werden kann, das an einer beliebigen Stelle platziert werden kann. 

\section{Idee}
Die Hauptidee unseres Konzepts besteht darin die kompakte Bauweise einer Dish-Stirling-Anlage und stationäre Stromerzeugung eines Parabolrinnenkraftwerks mit möglichst geringen Kosten zu kombinieren. Die Anlage soll dabei aus einem großen, fest montierten Hauptspiegel sowie einem auf einer Schiene beweglichen kleineren Zweitspiegel bestehen, was in Abbildung~\ref{pic:system_rendered} veranschaulicht wird.
\begin{figure}[htb]
	\centering
	\includegraphics[width=\textwidth]{images/SMALL_MIRROR}
	\caption[Gesamtsystem]{Gesamtsystem mit großem und kleinen Spiegel, schwenkbarer Führungsschiene und Absorber.}
	\label{pic:system_rendered}
\end{figure}
Das einfallende Sonnenlicht wird zunächst vom Hauptspiegel reflektiert. Der Zweitspiegel folgt stets dem Fokuspunkt des Hauptspiegels, welcher seine Position abhängig von der Tages- und Jahreszeit verändert, und reflektiert das Licht anschließend zu einem in den Hauptspiegel integrierten Absorber.  Die im Absorber gesammelte Energie wird zur Stromerzeugung in einem stationären Kraftwerk verwendet.
Neben dem geschildertem Aufbau des Kraftwerks, wird ein Konzept zur Energiespeicherung in einem Luftdrucksystem vorgestellt, welches vor allem für Privathaushalte eine On-Demand-Stromerzeugung ermöglichen soll.

\section{Vorstellung des Konzepts}
Es wurde zuerst untersucht wie sich schräg einfallende Strahlung auf den Fokuspunkt der Strahlen auswirkt. Für solche Versuche wurde ein Raytracing Programm in Matlab geschrieben, das beliebig viele Strahlen in einer beliebigen Richtung erzeugt, den Schnittpunkt mit einem Spiegel findet, die Reflexionsrichtung berechnet und alles grafisch darstellen kann. 
\begin{figure}[htb]
	\centering
	\includegraphics[width=\textwidth]{images/2d_gerade}
	\caption{Gerader Einfall auf Parabolspiegel}
	\label{pic:2dreflektion}
\end{figure}
Das Ergebnis ist in Abbildung~\ref{pic:2dreflektion} zu sehen. Während die Strahlen bei senkrechtem Lichteinfall genau im Fokuspunkt gebündelt werden, verschiebt sich dieser
bei schrägem Einfall nicht nur, die Strahlen werden nun auch nicht mehr so gut gebündelt, sondern fächern auf wie in Abbildung~\ref{pic:2dreflektion_schraeg} dargestellt. 
\begin{figure}[htb]
	\centering
	\includegraphics[width=\textwidth]{images/2d_schraeg_20_grad}
	\caption{Schräger Einfall auf Parabolspiegel (20$^{\circ}$ zur vertikalen Achse)}
	\label{pic:2dreflektion_schraeg}
\end{figure}

\subsection{Erhoffte Vorteile}
Es wird erhofft, dass der Konstruktionsaufwand durch einen stationären Hauptspiegel weitgehend gering ausfallen wird. Besonders soll hier hervorgehoben werden, dass der Hauptspiegel dadurch, dass er stationär ist viel einfacher die erforderlichen Windlasten aushält. Der Zweitspiegel kann sehr leicht gebaut werden und kann bei zu starkem Wind in den Schutz des Hauptspiegels gefahren werden. Die Motoren zur Nachführung des kleinen Spiegels können sehr klein ausgelegt werden. Unter dem Absorber soll direkt die Umwandlung in elektrischen Strom geschehen, was die Transportwege verkürzt und die Komplexität verringert. 

\section{Beschreibung des Programms}
Zur Berechnung der Geometrie des beschriebenen Zweispiegelsystems wurde ein Mat"-lab-Programm geschrieben. Dieses berechnet die optimalen Formen der Spiegel durch Lösen eines Maximierungsproblems, welches darin besteht, eine möglichst große Strahlendichte pro Absorberfläche zu erhalten. Nachfolgend wird die Vorgehensweise des Programms grob beschrieben.

\subsection*{Kollisionsdetektion}
Eine Kollision von Strahlen und Spiegelfläche muss zu vier Zeitpunkten des Programms überprüft werden:
\begin{enumerate}
\item Reflektion am großen Spiegel,
\item Reflektion am kleinen Spiegel (blockierende Rückseite),
\item Reflektion am kleinen Spiegel (spiegelnde Vorderseite),
\item Absorption in der Absorberfläche.
\end{enumerate}
Dabei wird jedes Mal auf dasselbe Subprogramm zugegriffen. Dieser "Dychotome Collision Tracker" benötigt einen Ausdruck einer Spiegelfunktion in der Form $Z = f(X,Y)$, sowie ein Array, in dem die Strahlen (Startpunkt, Richtung) gespeichert sind.
Der Collision Tracker folgt zunächst der Richtung jedes Strahls in einer festgelegten Schrittweite. Sobald eine Höhe $Z_{unter}$ erreicht wird, die kleiner als die Spiegelhöhe $Z_{Spiegel}$ an diesen Koordinaten ist, hat der Strahl den Spiegel durchstoßen.
Darauf beginnt die zweite Art der Suche, bei der der Suchraum mit jedem Schritt halbiert wird. Diese Dychotome Suche vergleicht, ob die Mitte zwischen zwei Randpunkten über oder unter dem Spiegel liegt. Ein Randpunkt ist dabei der soeben erreichte $Z_{unter}$, der zweite ist der Punkt $Z_{ueber}$, der vor dem Durchstoßen erreicht wurde. Zwischen diesen beiden Grenzen muss zwangsweise der genaue Kollisionspunkt liegen. Liegt die Mitte mit der Höhe $Z_{Mitte} = \dfrac{Z_{ueber} + Z_{unter}}{2}$ oberhalb des Spiegels an diesen Koordianten, so wird die Mitte die neue obere Grenze: $Z^{\ast}_{über} := Z_{Mitte}$. Andersherum wird die Mitte zur neuen unteren Grenze, wenn sie unterhalb des Spiegels liegt. Darauf wiederholt sich die Berechnung des Mittelpunkts und dessen Vergleich mit der Spiegelhöhe. Ist eine vorgegebene Genauigkeit des Unsicherheitsintervalls erreicht, terminiert der Algorithmus und gibt den aktuellen Mittelpunkt als Kollisionspunkt aus.

\subsection{Ablaufdiagramm (Top-Level)}
\subsection{Möglichkeiten und Grenzen}
 (versch. Breitengrade, über ein Jahr hinweg etc.)
\section{Gesamtaufbau}
\subsection{CAD-Modell mit allen Nebenaggregaten}
 (Antriebe: Stirling, Turbine)
\subsection{Verschiedene Generatorantriebe}
 (Stirling, Turbine)
\section{Kosten}

\section{Ergebnisse}
Bis zu diesem Zeitpunkt (\today) haben wir einige Szenarien der Optimierung durchgerechnet. Die grundlegende korrekte Funktion des Raytracers soll in Abbildung~\ref{pic:netteReflektion} verdeutlicht werden. Nur Sonnenstrahlen, die den blau dargestellten Absorber treffen, sind abgebildet. Der große und kleine Spiegel haben jeweils Parabolform , dabei wird die Grundfläche des großen Spiegels von den vorgegebenen \unit[10]{m$^2$} begrenzt.
\begin{figure}[htb]
	\centering
	\includegraphics[width=\textwidth]{images/netteReflektion}
	\caption[Bündelung schräg]{Schräger Einfall auf Parabolspiegel, Reflektion und Bündelung durch zweiten kleineren Spiegel.}
	\label{pic:netteReflektion}
\end{figure}
Zu erkennen ist die korrekte Verschiebung des kleinen Spiegels an den Ort des verschobenen Fokuspunktes (vgl. Abb.~\ref{pic:2dreflektion_schraeg}), sowie dessen Ausrichtung auf den Absorber.
Ungefähr \unit[40]{\%} der zur Verfügung stehenden großen Spiegelfläche werden von Strahlen getroffen, die den Absorber erreichen. Dies entspräche also einem Wirkungsgrad $\eta_{Sonne}$ von \unit[40]{\%}. Für eine Abschätzung des Gesamtwirkungsgrades $\eta_{gesamt}$ muss noch der Wirkungsgrad des nachgeschalteten Generators multipliziert werden.

\section{Alles Mögliche}

\begin{align*}
	AirMass(\xi)=\frac{1}{\cos(\xi)}
\end{align*}

\begin{figure}[htb]
	\centering
	\includegraphics[width=\textwidth]{images/Airmass.png}
	\caption{Luftmasse in Abhängigkeit vom Zenit-Winkel}
	\label{pic:AirMass}
\end{figure}

\begin{table}
\centering
	\caption{Werte für Strahlungsleistung. Wiki (engl) air mass solar energy. Stand 08.06.}
	\label{tab:airmass}
\begin{tabular}{|c|c|c|}
	\hline
	$\alpha$ & Air Mass & $\frac{W}{m^2}$\\
	\hline
	- & 0 & 1367\\
	\hline
	0 & 1 & 1040\\
	\hline
	23 & 1.09 & 1020\\
	\hline
	48.2 & 1.5 & 930\\
	\hline
	75 & 3.8 & 620\\
	\hline
	85 & 10 & 270\\
	\hline
\end{tabular}
\end{table}
Test
\begin{align*}
	I=1,1\cdot I_0 \cdot 0,7^{AM^{0,678}}
	\label{eqn:Intensity}
\end{align*}

\newpage
Wir sollten die Quellen mit BibTex organisieren, welches Programm nehmt ihr da? Ich kenn nur BibDesk aus dem IBT, werd das mal suchen.

Quellen:

\url{http://www.fvee.de/fileadmin/publikationen/Themenhefte/th2002/th2002_02_03.pdf}
\url{http://upload.wikimedia.org/wikipedia/commons/4/47/Carnot-eta.PNG}
\url{http://upload.wikimedia.org/wikipedia/commons/e/ed/EuroDishSBP_front.jpg}
\url{http://www.infiniacorp.com/index.html}


Preise für Stirling Motor:
\url{http://www.bhkw-prinz.de/senertec-dachs-stirling-se-mikro-kwk/1812#Preis}
\url{solo Stirling}
24.000 Euro sagt:
\url{http://energieberatung.ibs-hlk.de/planbhkw_stirling.htm}
Creative Commons Lizenz - Stirling
\url{http://ve-ingenieure.de/projekt_st05g_cnc.html}
endlich mal was professionelles:
\url{http://www.stirling.dk/index.php} und \url{http://www.stirling-energie.de/}
hm war leider wieder ohne Preis, dafür sagt der hier auch 25.000Euro
\url{http://www.stirlingmotor.org/downloads/basisinfo.pdf}

\end{document}